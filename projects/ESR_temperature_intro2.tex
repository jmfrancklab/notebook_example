\subsubsection{second round of VTU experiments with Braiman}
check frequency that we had used previously 9.33~GHz
looking at Br$_3$? using a SiO2 catalyst, since Al2O3 catalyst showed a defect in the EPR near g=2
this time, we want to go down to 100K, to see if we can quench relaxation mechanisms that could be broadening our signal??

hook up VTU
\o[13:37]{\textbf{note that it's really important to check the positioning of the transfer line before connecting it to avoid putting pressure on the dewar}}
set VTU to 200K, so that it's cold but not sucking up moisture (sample is stored on dry ice, so this should be the temperature that the sample is at)
if temperature doesn't drop, look for leakage at the bottom of the resonator, which would indicate the ground glass joint isn't attached properly
\o[13:53]{}
\begin{err}
    \o[13:40]{even though it sees the dewar is full, temperature is not moving}
    exit the program and start again
    \o[13:42]{doesn't help}
    \o[13:52]{turns out it wasn't hooked up}
\end{err}
insert sample and drop down to 100K and set settling time to 20~s (like last time)
\o[14:02]{started to slow down a bit, but at 112K}
\o[14:04]{at 102K}
go ahead and tune (near 9.33) when getting close
\o{$\checkmark$}
make a copy of \texttt{initial_scan_190K_10G_191108.DSC} for a wide scan with 10~G
modulation -- be sure to select all harmonics
\o[14:14]{}
let it run until we have decent SNR
\o[15:20]{at this point, have run \texttt{wide_10G_sample_run2_191125.DSC} and \texttt{wide_10G_sample_191125.DSC}}
make a copy of \texttt{zoom_left_191108}, make sure mod amp is 9.6~G, and select all harmonics and run that
\o[15:21]{}
\o[16:46]{done running two runs of signal for both zoom left and wide}
set it to warm up to 200K
\o[16:51]{it's warm}
pull out and switch background
\o[18:51]{at this point, took the background (called bg) and shut everything down (left cavity for Alec)}