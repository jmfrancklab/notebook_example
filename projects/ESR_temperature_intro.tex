\section{ESR Temperature control}
\subsection{VTU notes from JF}
\subsubsection{running VTU experiment with Braiman}
confusion about assembly of the dewar apparatus
\o{Braiman found manual for the ``VT ER4141'' on manualzz.com (Alec has link) -- verifies that there is no bottom teflon seal, only the rubber}
Insert 6mm o.d. 2mm (?) o.d. tube into VTU apparatus, measure Q and frequency
\o{9.338~GHz Q 2300}
Turn on the VT at a low temperature:
Turn on the evaporator heater at 10\%.
\o{$\checkmark$}
\begin{err}
    here, it's not clear what a good heater power is for different temperatures -- we should know this
    \o[12:13]{it does this automaticaly}
\end{err}
dry ice is -78.5°C -- so set 194 K
\o[12:13]{cools down to 192 K fairly quickly} so go back to 275 to wait for actual sample
put actual sample in (w/ Br and SiO$_2$
\o[12:25]{does seem like tuning gets easier once the temperature drops below CO$_2$, but this is qualitative}
allow the temperature to stabilize
\o[12:26]{good now}
\o[13:12]{for next time:}
set evaporator to 30\%, heater limit to 20\% and settling time to 60
\o[13:12]{(didn't do this)}
tune (from scratch)
\o[12:29]{diode starts to float at 6~dB}
\o[12:30]{tuned -- Q 6300 at 9.33 GHz}
copy \texttt{initial_scan_191101.DSC}
\o[12:32]{}
turn on all harmonics
\o[12:32]{}
\o[13:12]{finally stabilized temperature after much messing!}
\begin{err}
    \o[12:33]{temperature is too low! -- flow rate was very high before this}
    open the temperature control window and mess with things
    \o[12:36]{it set the evaporator to 0\%, which lead to a temperature that
    was too high -- I pushed it to 2\%. By pushing the temperature settling
    time to 7 s, I got it to stabilize}
    \o[12:45]{it's oscillating again!}
    turn down both heater and evaporator
    \o[12:51]{still problematic}
    increase the temperature settling time to 20~s to give it more time to
    stabilize
    \o[13:00]{nothing; notice that the gas flow coming out of the cavity seems very low}
    increase the gas flow to 20\%
    \o[13:05]{seems to be helping}
    to make life easier, set temperature to 191~K and set tolerance to 3~K
    \o[13:06]{it increased evaporator to 60\% and I increased the heater limit to 20\%}
    \o[13:07]{now it's stabilized}
    try to decrease evaporator to 50\% and see if it stays stable
    \o[13:08]{ok for a few seconds}
    and to 40\%
    \o{seemed OK}
    and to 30\%
    \o[13:10]{just year decrease in flow now}
\end{err}
run
\o[12:32]{}
\o[13:13]{at 3 scans, I think I want 4 times the SNR, so run to at least 64 scans}
\o[13:27]{at 22 scans now}
increase the modulation amplitude to to 10~G, and re-run 64 scans
\o[14:53]{this finished, possibly see more signal along large baseline slope but hard to tell}
put into standby, Braiman asked for more scans so tuning again (dip moved?)
(Q = 6700, f = 9.333 GHz)
\o[15:10]{pull out sample and warm up}




